\documentclass[12pt,a4paper]{report}

\usepackage[francais]{babel}
\usepackage[utf8]{inputenc}
\usepackage[T1]{fontenc}

\begin{document}

\chapter{Comparaison avec le projet AntSTO}

\section{Lisibilité}

Contrairement au notre, le code est peu commenté (en partie en allemand...) mais
aussi bien plus explicite.
L'organisation des fichiers est similaire dans les deux projets, les différents
rôles des modules ont été identifies de la même manière.

\section{Correction}

Le programme ne semble pas envoyer de requête (bien que ce soit implémenté) au
serveur et n'a donc pas le comportement attendu. Le notre envoie effectivement
les requêtes mais elle ne fonctionnement pas à cause des cookies. Accessoirement
la boucle de jeu a aussi été implémentée dans notre programme.\newline

Une large partie de l'API a été implémentée sans tenir compte d'un élément
important : la plupart des actions de l'API requiert qu'un utilisateur soit
connecté et cela n'a pas été traité dans leur module de communication.
Toutefois, le code étant bien organisé, il ne serait pas difficile
d'intégrer le cookie envoyé par le serveur dans les requêtes effectués.


\section{Robustesse}

Il est compliqué d'évaluer la capacité des traitements de données effectuées
par leur projet car cela n'a pas été implémenté. On peut voir qu'il y a un
début de traitement avec un module des codes d'erreurs du serveur et la
récupération du statut de réponse du serveur suite à un appel à l'API.
Toutefois, cette partie n'a pas pu être agréger au reste du projet donc
elle n'est pas testable.\newline

La manière par laquelle nous avons traité les réponses du serveur est de
s'assurer que le statut renvoyé correspond à un succès puis effectuer le
traitement correspondant à une action de l'API. Malgré le fait que nous ne
pouvions pas tester notre traitement des données à cause des cookies, nous
avons utilisé la documentation mise à notre disposition en début de projet
afin de nous assurer que le traitement effectué pour chaque appel à l'API
est correcte. Nous récupérions la réponse sous le format JSON pour le
copier dans un fichier et tester nos fonctions dessus. Ce travail n'est
pas visible que par l'implémentation que nous proposons des données
puisque les tests ont été effectués en dehors de ce projet.\newline

Pour conclure cette partie, on voit qu'un module a été mise en place
pour traiter les erreurs du serveur mais que le traitement des données
envoyés par le serveur n'a été effectué. Leur manière de traiter les erreurs
est beaucoup plus extensible et précise que la nôtre mais nous avons effectué
le traitement des données envoyés par le serveur contraitement à leur projet.

\section{Efficacité}

Il est difficile de juger (ni même d'observer) de l'efficacité des programmes,
d'une part parce qu'ils ne sont pas corrects et d'autre part parce c'est le
serveur qui donne le « rythme ».

En terme de consommation mémoire, dans les deux cas c'est principalement
l'interprétation des données reçues qui influe sur la consommation, le reste
étant du calcul fait à partir de ces données. Ces données étant les mêmes pour
les deux programmes, la consommation est, à peu de chose près, identique. 

\section{Généralisation}

Notre code permet la création de comportement des fourmis (et des zombies)
indépendamment du reste du programme (il suffit de donner la bonne valeur
dans le fichier qui organise la boucle de jeu). On peut également modifier
facilement les informations que contiennent les fourmis grâce a la
représentation donné. L'autre projet n'a pas été assez loin dans
l'implémentation de ces aspects du programme pour avoir cette flexibilité.\newline

Par contre leur module de communication est beaucoup plus facilement
modifiable pour suivre la potentiel évolution de l'API.

\end{document}
